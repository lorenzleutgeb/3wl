\documentclass{vutinfth} % Remove option 'final' to obtain debug information.

% Load packages to allow in- and output of non-ASCII characters.
\usepackage{lmodern}        % Use an extension of the original Computer Modern font to minimize the use of bitmapped letters.
\usepackage[T1]{fontenc}    % Determines font encoding of the output. Font packages have to be included before this line.
\usepackage[utf8]{inputenc} % Determines encoding of the input. All input files have to use UTF8 encoding.

% Extended LaTeX functionality is enables by including packages with \usepackage{...}.
\usepackage{fixltx2e}   % Provides fixes for several errors in LaTeX2e.
\usepackage{amsmath}    % Extended typesetting of mathematical expression.
\usepackage{amsthm}
\usepackage{amssymb}    % Provides a multitude of mathematical symbols.
\usepackage{mathtools}  % Further extensions of mathematical typesetting.
\usepackage{microtype}  % Small-scale typographic enhancements.
\usepackage{enumitem}   % User control over the layout of lists (itemize, enumerate, description).
\usepackage{multirow}   % Allows table elements to span several rows.
\usepackage{booktabs}   % Improves the typesettings of tables.
\usepackage{subcaption} % Allows the use of subfigures and enables their referencing.
\usepackage[ruled,linesnumbered,algochapter]{algorithm2e} % Enables the writing of pseudo code.
\usepackage[usenames,dvipsnames,table]{xcolor} % Allows the definition and use of colors. This package has to be included before tikz.
\usepackage{nag}       % Issues warnings when best practices in writing LaTeX documents are violated.
\usepackage{hyperref}  % Enables cross linking in the electronic document version. This package has to be included second to last.
\usepackage[acronym,toc]{glossaries} % Enables the generation of glossaries and lists fo acronyms. This package has to be included last.
\usepackage{tikz}
\usetikzlibrary{matrix}
\usetikzlibrary{shapes.multipart,calc}
\usetikzlibrary{positioning}
\usetikzlibrary{fit}

% Define convenience functions to use the author name and the thesis title in the PDF document properties.
\newcommand{\authorname}{Lorenz Leutgeb} % The author name without titles.
\newcommand{\thesistitle}{Efficient propagation for lazy-grounding Answer Set solving} % The title of the thesis. The English version should be used, if it exists.

\hypersetup{
    pdfpagelayout   = TwoPageRight,
    linkbordercolor = {Melon},
    pdfauthor       = {\authorname},
    pdftitle        = {\thesistitle},
    pdfsubject      = {Subject},              % The document's subject in the document properties (optional).
    pdfkeywords     = {asp, solver, propagation, nogood}
}

\setsecnumdepth{subsection} % Enumerate subsections.

\nonzeroparskip             % Create space between paragraphs (optional).
\setlength{\parindent}{0pt} % Remove paragraph identation (optional).

\makeindex      % Use an optional index.
\makeglossaries % Use an optional glossary.
%\glstocfalse   % Remove the glossaries from the table of contents.

% Set persons with 4 arguments:
%  {title before name}{name}{title after name}{gender}
%  where both titles are optional (i.e. can be given as empty brackets {}).
\setauthor{}{\authorname}{}{male}
\setadvisor{Prof.~Dr.}{Thomas Eiter}{}{male}

% For bachelor and master theses:
\setfirstassistant{Dr.}{Antonius Weinzierl}{}{male}
%\setsecondassistant{Pretitle}{Forename Surname}{Posttitle}{male}
%\setthirdassistant{Pretitle}{Forename Surname}{Posttitle}{male}

\setaddress{Engilgasse 3a, 1160 Wien}
\setregnumber{1127842}
\setdate{31}{10}{2016} % Set date with 3 arguments: {day}{month}{year}.
\settitle{\thesistitle}{Efficient propagation for lazy-grounding Answer Set solving} % Sets English and German version of the title (both can be English or German).
%\setsubtitle{Optional Subtitle of the Thesis}{Optionaler Untertitel der Arbeit} % Sets English and German version of the subtitle (both can be English or German).

\setthesis{bachelor}

\setcurriculum{Software \& Information Engineering}{Software \& Information Engineering} % Sets the English and German name of the curriculum.

% Define convenience macros.
\newcommand{\todo}[1]{{\color{red}\textbf{TODO: {#1}}}} % Comment for the final version, to raise errors.

\theoremstyle{example}
\newtheorem{example}{Example}[section]

\theoremstyle{definition}
\newtheorem{definition}{Definition}[section]

\theoremstyle{theorem}
\newtheorem{theorem}{Theorem}[section]

\theoremstyle{lemma}
\newtheorem{lemma}{Lemma}[theorem]

\theoremstyle{corollary}
\newtheorem{corollary}{Corollary}[theorem]

\newtheorem*{remark}{Remark}

\newcommand{\mbt}{must-be-true}
\newcommand{\negstrong}[1]{\overline{#1}^s}
\newcommand{\negweak}[1]{\overline{#1}^w}

% Function B transforming an assignment into a boolean assignment.
\newcommand{\bass}{\mathcal{B}}

% An assignment A.
\newcommand{\ass}{\mathbf{A}}

% Herbrand Base function of some logic program.
\newcommand{\hb}{\textit{HB}}

\newcommand{\bT}{\mathbf{T}}
\newcommand{\bM}{\mathbf{M}}
\newcommand{\bF}{\mathbf{F}}

\newcommand{\thrice}{{\{\bT, \bM, \bF \}}}

\newcommand{\wkn}{\textit{weaken}}

\usepackage{csquotes}

\begin{document}

\frontmatter % Switches to roman numbering.
% The structure of the thesis has to conform to
%  http://www.informatik.tuwien.ac.at/dekanat

\addtitlepage{naustrian}
\addtitlepage{english}
\addstatementpage

\begin{danksagung*}
\todo{Ihr Text hier.}
\end{danksagung*}

\begin{acknowledgements*}
\todo{Enter your text here.}
\end{acknowledgements*}

\begin{kurzfassung}
\todo{Ihr Text hier.}
\end{kurzfassung}

\begin{abstract}
\todo{Enter your text here.}
\end{abstract}

\selectlanguage{english}

\tableofcontents % Starred version, i.e., \tableofcontents*, removes the self-entry.

% Switch to arabic numbering and start the enumeration of chapters in the table of content.
\mainmatter

\chapter{Introduction}

Answer Set Programming (ASP) is a programming paradigm aimed at solving problems by means of declarative and logic programming based on Nonmonotonic Reasoning. 

\section{Motivation}

\section{Problem Statement}

\section{Aim of the Work}

\section{Methodological Approach}

\section{Structure of the Work} % 3 sentences

This is just a test.\cite{Gebser:2012:CAS:2228640.2228952}

\chapter{Preliminaries}
\label{chap:preliminaries}

\section{Answer Set Programming}

% TODO: Introduction to ASP and it's semantics.

\section{State of the Art in ASP Solving}

\subsection{Approaches based on Pre-Grounding}

% TODO: Explain how a solver works and what has been done already (grounding on the fly, omiga, asperix).
% TODO: What is a nogood? What is propagation?

% Everything that later sections build up on. Quote a lot. This chapter can be long. Show understanding of the matter in own words (i.e. what is an answer set incl. example?).

\subsection{Approaches based on Lazy Grounding}

\section{Analysis}

\section{Comparison and Summary of Existing Approaches}

\todo{Enter your text here.}

\chapter{Propagation for lazy-grounding Answer Set solving}

This chapter contextualizes propagation by providing definitions extending chapter \ref{chap:preliminaries}. It then presents and thoroughly explains the algorithms and data structures involved in propagation and concludes in a proof of soundness and completeness thereof.

\section{Definitions}

% Signed literal vs. boolean signed literal.

Following definitions lay out the vocabulary and concepts towards an explanation of unit propagation. Most of them fundamentally depend on \emph{atoms} (usually denoted $v$) but do not refer to neither a set of atoms nor their domain. This is because all definitions are bound to a logic program $P$ to be solved (i.e.~to find stable models for). The set of ground atoms wrt.~$P$, effectively its Herbrand Base $\textit{HB}(P)$, is obtained by the grounding process, which is not detailed in this work. What might seem to be lacking from the definitions therefore is the context of the input program $P$ and therefore the domain of atoms. This context is considered implicit in the following.

% TODO: Where does v come from? Should be in some HB(P)?
\begin{definition}
A \emph{ground atom} (or simply \emph{atom}) wrt.~a logic program $P$ is an element of $\textit{HB}(P)$, usually denoted $v$.
\end{definition}

\begin{definition}
A \emph{signed literal} $\sigma$ of the form $\mathbf{T}v$, $\mathbf{M}v$ or $\mathbf{F}v$ where $v$ is an atom and $\mathbf{T}v$ expresses that $v$ is \emph{true}, $\mathbf{M}v$ that it \emph{must be true}, and $\mathbf{F}v$ that it is \emph{false}.
\end{definition}

\begin{definition}
A \emph{boolean signed literal} $\sigma$ of the form $\mathbf{T}v$ or $\mathbf{F}v$ where $v$ is an atom and  $\mathbf{T}v$ expresses that $v$ is \emph{true}, and $\mathbf{F}v$ that it is \emph{false}.
\end{definition}

\begin{definition}
The function $\wkn(\sigma)$ takes a boolean signed literal $\sigma$ and transforms it into a signed literal referred to as its \emph{weak form}, meaning that while \emph{false} stays \emph{false}, e.g.~$\wkn(\bF v) = \bF v$, \emph{true} is mapped to \emph{\mbt}, e.g.~$\wkn(\bT v) = \bM v$.
\end{definition}

% TODO: Remark on boolean signed literal vs. signed literal with context of clasp?

\begin{definition}
An \emph{assignment} $\ass$ is a sequence $(\sigma_1, \ldots, \sigma_n)$ of signed literals $\sigma_i$ of the form $\star v_i$ with $\star \in \thrice$ where $v_i \not = v_j$ for $1 \leq i < j \leq n$, or \emph{conflict}.
\end{definition}

\begin{remark}
Below, assignments are sometimes also used as sets, in which case the set represented by some assignment is simply the set of all signed literals contained in the sequence.
\end{remark}

\begin{definition}
For every assignment, a respective \emph{boolean assignment}, denoted $\bass(\ass)$ can be constructed by collapsing all atoms that \emph{must be true} to being \emph{true}:$$\bass(\ass) = \{ \sigma \ | \ \sigma \in \ass, \sigma = \mathbf{T}v \textrm{ or } \sigma = \mathbf{F}v \} \cup \{ \sigma' \ | \ \sigma \in \ass, \sigma = \mathbf{M}v, \sigma' = \mathbf{T}v \}$$
\end{definition}

\begin{definition}
Strong complement, denoted by $\negstrong{\sigma}$, and weak complement, $\negweak{\sigma}$, mapping $\mathbf{F}v$ to $\mathbf{T}v$ and $\mathbf{F}v$ to $\mathbf{M}v$ respectively, of a signed literal are defined by the following truth table:%equalities: $\negstrong{\mathbf{T}v} = \mathbf{F}v$, $\negstrong{\mathbf{M}v} = \mathbf{F}v$ and  $\negstrong{\mathbf{F}v} = \mathbf{T}v$, while $\negweak{\mathbf{T}v} = \mathbf{F}v$, $\negweak{\mathbf{M}v} = \mathbf{F}v$ and $\negweak{\mathbf{F}v} = \mathbf{M}v$.
% "andere Richtung auch erklären"

\begin{center}
\begin{tabular}{|c|cc|}
\hline
$\sigma$&$\negstrong{\sigma}$&$\negweak{\sigma}$\\
\hline
\hline
$\mathbf{T}v$&$\mathbf{F}v$&$\mathbf{F}v$\\
$\mathbf{M}v$&$\mathbf{F}v$&$\mathbf{F}v$\\
$\mathbf{F}v$&$\mathbf{T}v$&$\mathbf{M}v$\\
\hline
\end{tabular}
\end{center}
\end{definition}

The assignment obtained by appending the literal $\sigma$ to $\ass$ is denoted by $\ass \circ \sigma$. In case $\negstrong{\sigma} \in \ass \vee \negweak{\sigma} \in \ass$ however, $\ass \circ \sigma =$ \emph{conflict} to indicate that appending $\sigma$ conflicts with other literals in $\ass$. Furthermore $\emph{conflict} \circ \sigma = \emph{conflict}$. To ensure that the assignment contains at most one literal per atom $v$, in case $\mathbf{M}v \in \ass$, then $\mathbf{M}v$ is removed when before $\mathbf{T}v$.

\begin{definition}
An atom $v$ is \emph{unassigned} with respect to some assignment $\ass$ iff $$\ass \cap \{\bT v, \bM v, \bF v \} = \emptyset$$
\end{definition}

\begin{definition}
A signed literal $\star v$ with $\star \in \thrice$ is \emph{unassigned} wrt.~$\ass$ if $v$ is unassigned wrt.~$\ass$.
\end{definition}

\begin{definition}
A \emph{nogood} reflects a partial assignment that cannot be extended to a solution. Here, a \emph{nogood} is a set $\{ \sigma_1, \ldots, \sigma_n \}$ of signed literals.
\end{definition}

\begin{definition}
A nogood $\delta$ is \emph{violated} by an assignment $\mathbf{A}$ if $\delta \subseteq \bass(\ass)$.
\end{definition}

\begin{definition}
A nogood $\delta$ is \emph{satisfied} wrt.~some assignment $\ass$ iff there is no $\ass' \supseteq \ass$ s.t.~$\delta$ is violated under $\ass'$.
\end{definition}

\begin{definition}
A nogood is \emph{strictly unit} with respect to some assignment $\mathbf{A}$ iff it contains exactly one \emph{unassigned} literal and all other literals are elements of $\mathbf{A}$.
\end{definition}

\begin{definition}
A nogood is \emph{weakly unit} with respect to some assignment $\mathbf{A}$ iff it contains exactly one \emph{unassigned} literal and all other literals are \emph{weak} elements of $\mathbf{A}$.
\end{definition}

%To efficiently retrieve the set of nogoods that need to be checked for unity, a data structure of watches was 

Watches are a set $\Delta = \{ (v_1, W_1), \ldots, (v_n, W_n) \}$ of pairs $(v_i, W_i)$, $1 \leq i \leq n$, containing an atom $v_i$ and a triple $W_i$, referred to as the \emph{watches for $v_i$}, consisting of $(W_i^\mathbf{T}, W_i^\mathbf{M}, W_i^\mathbf{F})$, three sets of nogoods that potentially propagate in case $v$ is assigned to $\mathbf{T}$, $\mathbf{M}$ or $\mathbf{F}$ respectively. Also there are no two pairs $(v, W), (v', W') \in \Delta$ s.t.~$v = v'$, i.e.~$\Delta$ is a function mapping an atom to a triple of its watches. We write $\Delta(v) = W$ to denote $(v, W) \in \Delta$. Furthermore, let $\Delta(v, \star) = W^\star$ for $\star \in \thrice$ and $\Delta(v) = (W^\mathbf{T}, W^\mathbf{M}, W^\mathbf{F})$ and let $\Delta(\sigma) = \Delta(v, \star)$ for a signed literal $\sigma = \star v$.

\begin{figure}[h]
  \centering
\begin{tikzpicture}[stack/.style={rectangle split, rectangle split parts=#1,draw, anchor=center},->]
\node(s)[stack=4]  {
                 Atom     % text
\nodepart{two}   $v_1$     % two
\nodepart{three} $v_2$      % three
\nodepart{four}  \vdots % four
};

\node(v1t)[stack=3, rectangle split horizontal, above right=2cm and 3cm of s.two] {
$\delta_1$     % two
\nodepart{two} $\delta_2$      % three
\nodepart{three}  \ldots % four
};

\node(v1m)[stack=3, rectangle split horizontal, above right=1cm and 3.5cm of s.two] {
$\delta_1$     % two
\nodepart{two} $\delta_2$      % three
\nodepart{three}  \ldots % four
};

\node(v1f)[stack=3, rectangle split horizontal, above right=0cm and 4cm of s.two] {
$\delta_1$     % two
\nodepart{two} $\delta_2$      % three
\nodepart{three}  \ldots % four
};

\path (s.two east)
edge [out=east,in=west, left] node {$\bT$} (v1t)
edge [out=east,in=west, above] node {$\bM$} (v1m)
edge [out=east,in=west, above] node {$\bF$} (v1f)
;

\end{tikzpicture}
  \caption{The layout of the \emph{watches} data structure.}
  \label{fig:watches} % \label has to be placed AFTER \caption (or \subcaption) to produce correct cross-references.
\end{figure}

\section{Unit Propagation}

\begin{algorithm}
  \SetKw{BreakFor}{break for}
  \KwIn{An assignment $\mathbf{A} = (\sigma_1, \ldots, \sigma_n)$,
        some~reference~to~an~element~$i$~with~$1 \leq i < n$ and
        a~set~of~watched~nogoods~$\Delta$.}
  \KwOut{The assignment $\mathbf{A}$, extended by means of unit propagation and $\Delta$ with updated watches.}
  \While{$i \leq n$}
  {
    $\sigma_i \leftarrow \mathbf{A}[i]$\\
    \uIf{$\sigma_i$ is of $\mathbf{M}v$ or $\sigma_i$ is of $\mathbf{F}v$}
    {
      $\mathbf{A} \ \circ$ \textsc{UnitPropagationMBT}($\mathbf{A}$, $\sigma_i$, $\Delta$)
    }
    \Else
    {
      $\mathbf{A} \ \circ$ \textsc{UnitPropagationMBT}($\mathbf{A}$, $\wkn(\sigma_i)$, $\Delta$)\\
      $\mathbf{A} \ \circ$ \textsc{UnitPropagationTrue}($\mathbf{A}$, $\sigma_i$, $\Delta$)
    }
    $i \leftarrow i + 1$
  }
  \Return{$(\mathbf{A}, \Delta)$}
  \caption{\textsc{UnitPropagation}}
  \label{alg:prop}
\end{algorithm}

\begin{algorithm}
  \KwIn{An~assignment~$\mathbf{A}$,
        a~literal~$\sigma \in \mathbf{A}$,~and
        a~set~of~watched~nogoods~$\Delta$.}
  \KwOut{$\mathbf{A}$ extended by means of nogood propagation and $\Delta$ with updated watches.}
%  \ForEach{$\delta \in w_{=2}(\Delta, \sigma)$ of form $\{ \sigma, \sigma' \}$}
%  {
%    \If{$\sigma'$ is unassigned in $\mathbf{A}$}
%    {
%      $\mathbf{A} \leftarrow \mathbf{A} \circ \negweak{\sigma'}$
%    }
%  }
  \ForEach{$\delta \in \Delta(\sigma)$}
  {
    \uIf{$\delta$ is violated}
    {
      \Return{$(\text{conflict}, \Delta)$}
    }
    \uElseIf{$\delta$ is weakly unit with $\sigma'$ unassigned}
    {
      $\mathbf{A} \leftarrow \mathbf{A} \circ \negweak{\sigma'}$
    }
    \Else%If{there is some unassigned $\sigma' \in \delta$}
    {
      $\Delta(\sigma) \leftarrow \Delta(\sigma) \setminus \{ \delta \}$\\
%      Let $\sigma'$, $\sigma''$ be two unassigned literals in $\delta$.\\
      \ForEach{$\sigma'$, one of two unassigned literals in $\delta$  }
      {
        $\Delta(\sigma') \leftarrow \Delta(\wkn(\sigma)) \cup \{ \delta \}$\\
      }
%      \uIf{$\sigma'$ is of $\mathbf{T}v$}
%      {
%        $\Delta(\sigma') \leftarrow \Delta(\mathbf{M}v) \cup \{ \delta \}$\\
%      }
%      \Else
%      {
%        $\Delta(\sigma') \leftarrow \Delta(\sigma') \cup \{ \delta \}$\\
%      }
%      \uIf{$\sigma''$ is of $\mathbf{T}v$}
%      {
%        $\Delta(\sigma'') \leftarrow \Delta(\mathbf{M}v) \cup \{ \delta \}$\\
%      }
%      \Else
%      {
%        $\Delta(\sigma'') \leftarrow \Delta(\sigma'') \cup \{ \delta \}$\\
%      }
    }
  }
  \Return{$(\mathbf{A}, \Delta)$}
  \caption{\textsc{UnitPropagationMBT}}
  \label{alg:prop-mbt}
\end{algorithm}

Algorithm \ref{alg:prop-mbt} exhibits how unit propagation is used to infer assignments (of either false or \mbt) from watched nogoods. 
In line 1, nogoods $\Delta(\sigma)$ that are watched for assignments of $\sigma$ are resolved and bound to $\delta$. Each of these nogoods can now be in one of three distinct states:
\begin{enumerate}
\item It could be violated, leading to a conflicting assignment, or
\item it could be weakly unit, in which case a new assignment can be inferred from the nogood, or
\item in any other case, there must be at least two unassigned literals in $\delta$, which are to be watched for changes in assignments.
% TODO: Does not (1.) and not (2.) guarantee that there are two unassigned literals?! Nogood could be satisfied.
\end{enumerate}

For the possibility that with the assignment of $\sigma$ all literals in $\delta$ were assigned and the nogood is now violated (1.), the algorithm returns a conflicting assignment in line 3.

In case (2.) handled in line 5, $\ass$ is updated by appending the weak complement of $\sigma$. All literals in $\delta$ are assigned and weakly contained in $\ass$, therefore the only way that $\delta$ cannot be violated under $\ass$ is by appending the (weak) complement of $\sigma$ to $\ass$.



% TODO pointers need to be more explicit. E.g. where does \sigma' come from in line 11?!

\begin{algorithm}
  \KwIn{An assignment $\ass$,
        a literal $\sigma \in \ass$, and
        a set of watched nogoods~$\Delta$.}
  \KwOut{$\ass$ extended by means of nogood propagation.}
%  \ForEach{nogood $\delta \in w_{=2}(\Delta)$ of form $\{ \sigma, \sigma' \}$}
%  {
%    \If{$\sigma'$ is unassigned}
%    {
%      $\mathbf{A} \circ \negstrong{\sigma'}$
%    }
%  }
  \ForEach{nogood $\delta \in w(\Delta)$}
  {
  % L statt Theta
    $L \leftarrow \{ \sigma \ | \ \sigma \in \delta, \sigma \not \in \ass \text{, and } \sigma \ \text{is of form} \ \mathbf{T}v \} \setminus \{ \sigma, \text{head}(\delta) \} $\\
    \uIf{$L \not = \emptyset$}
    {
      Choose $\sigma' \in L$ s.t.~there is no $\sigma'' \in L$ with $\ass = (\ldots, \sigma', \ldots, \sigma'', \ldots)$.\\
      $\Delta(\sigma) \leftarrow \Delta(\sigma) \setminus \{ \delta \}$\\
      $\Delta(\sigma') \leftarrow \Delta(\sigma') \cup \{ \delta \}$\\
    }
    %\uElseIf{$\delta$ is violated}
    %{
    %  \Return{$(\text{conflict}, \Delta)$}
    %}
	\ElseIf{$\delta$ is strictly unit}
	{
	  $\mathbf{A} \circ \negstrong{\sigma'}$
	}
  }
  \Return{$(\mathbf{A}, \Delta)$}
  \caption{\textsc{UnitPropagationTrue}}
  \label{alg:prop-true}
\end{algorithm}

%In order to implement efficient propagation, a data

% No Unfounded Set checks -> MBT.

% Explain 2WL with MBT, propagation with MBT (pseudocode of propagation algorithm).

\chapter{Implementation}

\section{Benchmarks}

\chapter{Conclusion}

\section{Related Work}

\section{Conclusion}

\subsection{Related Work}

\backmatter

% Use an optional list of figures.
\listoffigures % Starred version, i.e., \listoffigures*, removes the toc entry.

% Use an optional list of tables.
\listoftables % Starred version, i.e., \listoftables*, removes the toc entry.

% Use an optional list of alogrithms.
\listofalgorithms
\addcontentsline{toc}{chapter}{List of Algorithms}

% Add an index.
\printindex

% Add a glossary.
\printglossaries

% Add a bibliography.
\bibliographystyle{alpha}
\bibliography{thesis}

\end{document}
