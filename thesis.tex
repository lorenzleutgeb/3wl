\documentclass{vutinfth} % Remove option 'final' to obtain debug information.

% Load packages to allow in- and output of non-ASCII characters.
\usepackage{lmodern}        % Use an extension of the original Computer Modern font to minimize the use of bitmapped letters.
\usepackage[T1]{fontenc}    % Determines font encoding of the output. Font packages have to be included before this line.
\usepackage[utf8]{inputenc} % Determines encoding of the input. All input files have to use UTF8 encoding.

% Extended LaTeX functionality is enables by including packages with \usepackage{...}.
\usepackage{fixltx2e}   % Provides fixes for several errors in LaTeX2e.
\usepackage{amsmath}    % Extended typesetting of mathematical expression.
\usepackage{amssymb}    % Provides a multitude of mathematical symbols.
\usepackage{mathtools}  % Further extensions of mathematical typesetting.
\usepackage{microtype}  % Small-scale typographic enhancements.
\usepackage{enumitem}   % User control over the layout of lists (itemize, enumerate, description).
\usepackage{multirow}   % Allows table elements to span several rows.
\usepackage{booktabs}   % Improves the typesettings of tables.
\usepackage{subcaption} % Allows the use of subfigures and enables their referencing.
\usepackage[ruled,linesnumbered,algochapter]{algorithm2e} % Enables the writing of pseudo code.
\usepackage[usenames,dvipsnames,table]{xcolor} % Allows the definition and use of colors. This package has to be included before tikz.
\usepackage{nag}       % Issues warnings when best practices in writing LaTeX documents are violated.
\usepackage{hyperref}  % Enables cross linking in the electronic document version. This package has to be included second to last.
\usepackage[acronym,toc]{glossaries} % Enables the generation of glossaries and lists fo acronyms. This package has to be included last.

% Define convenience functions to use the author name and the thesis title in the PDF document properties.
\newcommand{\authorname}{Lorenz Leutgeb} % The author name without titles.
\newcommand{\thesistitle}{Efficient propagation for lazy-grounding Answer Set solving} % The title of the thesis. The English version should be used, if it exists.

\hypersetup{
    pdfpagelayout   = TwoPageRight,
    linkbordercolor = {Melon},
    pdfauthor       = {\authorname},
    pdftitle        = {\thesistitle},
    pdfsubject      = {Subject},              % The document's subject in the document properties (optional).
    pdfkeywords     = {asp, solver, propagation, nogood}
}

\setsecnumdepth{subsection} % Enumerate subsections.

\nonzeroparskip             % Create space between paragraphs (optional).
\setlength{\parindent}{0pt} % Remove paragraph identation (optional).

\makeindex      % Use an optional index.
\makeglossaries % Use an optional glossary.
%\glstocfalse   % Remove the glossaries from the table of contents.

% Set persons with 4 arguments:
%  {title before name}{name}{title after name}{gender}
%  where both titles are optional (i.e. can be given as empty brackets {}).
\setauthor{}{\authorname}{}{male}
\setadvisor{Dr.}{Antonius Weinzierl}{}{male}

% For bachelor and master theses:
%\setfirstassistant{Pretitle}{Forename Surname}{Posttitle}{male}
%\setsecondassistant{Pretitle}{Forename Surname}{Posttitle}{male}
%\setthirdassistant{Pretitle}{Forename Surname}{Posttitle}{male}

\setaddress{Engilgasse 3a, 1160 Wien}
\setregnumber{1127842}
\setdate{31}{10}{2016} % Set date with 3 arguments: {day}{month}{year}.
\settitle{\thesistitle}{Efficient propagation for lazy-grounding Answer Set solving} % Sets English and German version of the title (both can be English or German).
%\setsubtitle{Optional Subtitle of the Thesis}{Optionaler Untertitel der Arbeit} % Sets English and German version of the subtitle (both can be English or German).

\setthesis{bachelor}

\setcurriculum{Software \& Information Engineering}{Software \& Information Engineering} % Sets the English and German name of the curriculum.

% Define convenience macros.
\newcommand{\todo}[1]{{\color{red}\textbf{TODO: {#1}}}} % Comment for the final version, to raise errors.

\newcommand{\mbt}{must-be-true}
\newcommand{\negstrong}[1]{\overline{#1}^s}
\newcommand{\negweak}[1]{\overline{#1}^w}

\usepackage{csquotes}

\begin{document}

\frontmatter % Switches to roman numbering.
% The structure of the thesis has to conform to
%  http://www.informatik.tuwien.ac.at/dekanat

\addtitlepage{naustrian}
\addtitlepage{english}
\addstatementpage

\begin{danksagung*}
\todo{Ihr Text hier.}
\end{danksagung*}

\begin{acknowledgements*}
\todo{Enter your text here.}
\end{acknowledgements*}

\begin{kurzfassung}
\todo{Ihr Text hier.}
\end{kurzfassung}

\begin{abstract}
\todo{Enter your text here.}
\end{abstract}

\selectlanguage{english}

\tableofcontents % Starred version, i.e., \tableofcontents*, removes the self-entry.

% Switch to arabic numbering and start the enumeration of chapters in the table of content.
\mainmatter

\chapter{Introduction}

\section{Motivation}

\section{Problem Statement}

\section{Aim of the Work}

\section{Methodological Approach}

\section{Structure of the Work} % 3 sentences

This is just a test.\cite{Gebser:2012:CAS:2228640.2228952}

\chapter{Preliminaries}

\section{Answer Set Programming}

% TODO: Introduction to ASP and it's semantics.

\section{State of the Art in ASP Solving}

\subsection{Approaches based on Pre-Grounding}

% TODO: Explain how a solver works and what has been done already (grounding on the fly, omiga, asperix).
% TODO: What is a nogood? What is propagation?

% Everything that later sections build up on. Quote a lot. This chapter can be long. Show understanding of the matter in own words (i.e. what is an answer set incl. example?).

\subsection{Approaches based on Lazy Grounding}

\section{Analysis}

\section{Comparison and Summary of Existing Approaches}

\todo{Enter your text here.}

\chapter{Propagation for lazy-grounding Answer Set solving}

A boolean \emph{assignment} $\mathbf{A}$ is a sequence $(\sigma_1, \ldots, \sigma_n)$ of signed literals $\sigma_i$ of the form $\mathbf{T}v_i$, $\mathbf{M}v_i$ or $\mathbf{F}v_i$ where $v_i \not = v_j$ for $i < j \leq n$, or \emph{conflict}.
A literal $\mathbf{T}v$ expresses that $v$ is \emph{true}, $\mathbf{M}v$ that it \emph{must be true}, and $\mathbf{F}v$ that it is \emph{false}. Below, assignments are sometimes also used as sets, in which case the set represented by some assignment is simply the set of all signed literals it contains.

Strong negation, denoted by $\negstrong{\sigma}$, and weak negation, $\negweak{\sigma}$, mapping $\mathbf{F}v$ to $\mathbf{T}v$ and $\mathbf{F}v$ to $\mathbf{W}v$ respectively, of a signed literal are defined by the following truth table:%equalities: $\negstrong{\mathbf{T}v} = \mathbf{F}v$, $\negstrong{\mathbf{M}v} = \mathbf{F}v$ and  $\negstrong{\mathbf{F}v} = \mathbf{T}v$, while $\negweak{\mathbf{T}v} = \mathbf{F}v$, $\negweak{\mathbf{M}v} = \mathbf{F}v$ and $\negweak{\mathbf{F}v} = \mathbf{M}v$.

\begin{center}
\begin{tabular}{|c|cc|}
\hline
$\sigma$&$\negstrong{\sigma}$&$\negweak{\sigma}$\\
\hline
\hline
$\mathbf{T}v$&$\mathbf{F}v$&$\mathbf{F}v$\\
$\mathbf{M}v$&$\mathbf{F}v$&$\mathbf{F}v$\\
$\mathbf{F}v$&$\mathbf{T}v$&$\mathbf{M}v$\\
\hline
\end{tabular}
\end{center}

The sequence obtained by appending the literal $\sigma$ to $\mathbf{A}$ is denoted by $\mathbf{A} \circ \sigma$. In case $\negstrong{\sigma} \in \mathbf{A} \vee \negweak{\sigma} \in \mathbf{A}$ however, $\mathbf{A} \circ \sigma =$ \emph{conflict} to indicate that appending $\sigma$ conflicts with other literals in $\mathbf{A}$. Furthermore $\emph{conflict} \circ \sigma = \emph{conflict}$.

A literal $\mathbf{X}v$ with $\mathbf{X} \in \{ \mathbf{T}, \mathbf{M}, \mathbf{F} \}$ is \emph{unassigned} with respect to some assignment $\mathbf{A}$ iff $\mathbf{T}v \not \in \mathbf{A} \wedge \mathbf{M}v \not \in \mathbf{A} \wedge \mathbf{F}v \not \in \mathbf{A}$.

An atom $v$ is \emph{unassigned} with respect to some assignment $\mathbf{A}$ iff $\mathbf{T}v \not \in \mathbf{A} \wedge \mathbf{M}v \not \in \mathbf{A} \wedge \mathbf{F}v \not \in \mathbf{A}$.

A nogood reflects a partial assignment that cannot be extended to a solution. Here, a \emph{nogood} is a set $\{ \sigma_1, \ldots, \sigma_n \}$ of signed literals. A nogood $\delta$ is \emph{violated} by an assignment $\mathbf{A}$ if $\delta \subseteq \mathbf{A}$.

A nogood is \emph{strictly unit} with respect to some assignment $\mathbf{A}$ iff it contains exactly one \emph{unassigned} literal and all other literals are elements of $\mathbf{A}$.

A nogood is \emph{weakly unit} with respect to some assignment $\mathbf{A}$ iff it contains exactly one \emph{unassigned} literal and all other literals are \emph{weak} elements of $\mathbf{A}$.

%To efficiently retrieve the set of nogoods that need to be checked for unity, a data structure of watches was 

Watches are triples of an atom and two \enquote{watch structures} $(v, W_{=2}, W_{>2})$, with $W_{=2}$ and $W_{>2}$ consisting of $(W_\mathbf{T}, W_\mathbf{M}, W_\mathbf{F})$,  three sets of nogoods that potentially propagate in case $v$ is assigned to $\mathbf{T}$, $\mathbf{M}$ or $\mathbf{F}$ respectively.

\begin{figure}[h]
  \centering
  \includegraphics[width=\textwidth]{watches}
  \caption{The layout of the \emph{watches} data structure.}
  \label{fig:watches} % \label has to be placed AFTER \caption (or \subcaption) to produce correct cross-references.
\end{figure}

\section{Unit Propagation}

\begin{algorithm}
  \SetKw{BreakFor}{break for}
  \KwIn{An assignment $\mathbf{A} = (\sigma_1, \ldots, \sigma_n)$,
        some~reference~to~an~element~$i$~with~$1 \leq i < n$ and
        a~set~of~watched~nogoods~$\Delta$.}
  \KwOut{The assignment $\mathbf{A}$, extended by means of unit propagation and $\Delta$ with updated watches.}
  \While{$i \leq n$}
  {
    $\sigma_i \leftarrow \mathbf{A}[i]$\\
    \uIf{$\sigma_i$ is of $\mathbf{M}v$ or $\sigma_i$ is of $\mathbf{F}v$}
    {
      $\mathbf{A} \ \circ$ \textsc{UnitPropagationMBT}($\mathbf{A}$, $\sigma_i$, $\Delta$)
    }
    \Else
    {
      $\mathbf{A} \ \circ$ \textsc{UnitPropagationMBT}($\mathbf{A}$, $\mathbf{M}v$, $\Delta$)\\
      $\mathbf{A} \ \circ$ \textsc{UnitPropagationTrue}($\mathbf{A}$, $\sigma_i$, $\Delta$)
    }
    $i \leftarrow i + 1$
  }
  \Return{$(\mathbf{A}, \Delta)$}
  \caption{\textsc{UnitPropagation}}
  \label{alg:prop}
\end{algorithm}

\begin{algorithm}
  \KwIn{An~assignment~$\mathbf{A}$,
        a~literal~$\sigma \in \mathbf{A}$,~and
        a~set~of~watched~nogoods~$\Delta$.}
  \KwOut{$\mathbf{A}$ extended by means of nogood propagation and $\Delta$ with updated watches.}
  \ForEach{$\delta \in w_{=2}(\Delta, \sigma)$ of form $\{ \sigma, \sigma' \}$}
  {
    \If{$\sigma'$ is unassigned in $\mathbf{A}$}
    {
      $\mathbf{A} \leftarrow \mathbf{A} \circ \negweak{\sigma'}$
    }
  }
  \ForEach{$\delta \in w_{>2}(\Delta, \sigma)$}
  {
    \uIf{$\delta$ is weakly unit with $\sigma'$ unassigned}
    {
      $\mathbf{A} \leftarrow \mathbf{A} \circ \negweak{\sigma'}$
    }
    \uElseIf{$\delta$ is violated}
    {
      \Return{$(\text{conflict}, \Delta)$}
    }
    \Else%If{there is some unassigned $\sigma' \in \delta$}
    {
      Let $\sigma'$, $\sigma''$ be two unassigned literals in $\delta$.\\
      Point watch pointers at $\sigma'$ and $\sigma''$\\
      
      $i \leftarrow $ index of $\sigma'$, the other watched literal\\
      $j \leftarrow $ index of unassigned literal, s.t.~$i \not = j$\\
      Repoint pointer pointing at $\sigma$ to $j$.\\

      Remove $\delta$ from watchlist for $\sigma$ and
      add it to watches for $\delta[j]$.
    }
  }
  \Return{$(\mathbf{A}, \Delta)$}
  \caption{\textsc{UnitPropagationMBT}}
  \label{alg:prop-mbt}
\end{algorithm}

Algorithm \ref{alg:prop-mbt} exhibits how unit propagation is used to infer assignments (of either false or \mbt) from watched nogoods. In lines 1-4 propagation of binary nogoods are handled: For every nogood that contains $\sigma$ and exactly one other literal $\sigma'$, propagation is trivial. If $\sigma'$ is unassigned it is assigned to its weak-negated form.
Lines 6-13 cover the more sophisticated general case for nogoods with more than two elements. The nogood is scanned for unassigned literals, as in this case (check in line 7) it cannot be unit, and the pointer pointing at $\sigma$ is moved to/pointed at the unassigned literal $\sigma'$ (line 8). Watches are readjusted to reflect pointer movement in line 9. If there is no unassigned literal (lines 10 to 12), the nogood propagates in case it is weakly unit. $\sigma'$ is assigned to it's weak-negated form.

% TODO pointers need to be more explicit. E.g. where does \sigma' come from in line 11?!

\begin{algorithm}
  \KwIn{An assignment $\mathbf{A}$,
        a literal $\sigma \in \mathbf{A}$, and
        a set of watched nogoods~$\Delta$.}
  \KwOut{$\mathbf{A}$ extended by means of nogood propagation.}
  \ForEach{nogood $\delta \in w_{=2}(\Delta)$ of form $\{ \sigma, \sigma' \}$}
  {
    \If{$\sigma'$ is unassigned}
    {
      $\mathbf{A} \circ \negstrong{\sigma'}$
    }
  }
  \ForEach{nogood $\delta \in w_{>2}(\Delta)$ with $\sigma \in \delta$}
  {
    $X \leftarrow \{ x \ | \ x \in \delta \wedge x \ \text{is of form} \ \mathbf{T}v \} \setminus \{ \sigma, \text{head}(\delta) \} \setminus \mathbf{A} $\\
    \uIf{$X \not = \emptyset$}
    {
      Choose $\sigma'$ from $X$ by highest decision level.\\
      Repoint pointer from $\sigma$ to $\sigma'$.\\
      Remove $\delta$ from watches for $\sigma$ and add it to watches for $\sigma'$.\\
    }
    %\uElseIf{$\delta$ is violated}
    %{
    %  \Return{$(\text{conflict}, \Delta)$}
    %}
	\ElseIf{$\delta$ is strictly unit}
	{
	  $\mathbf{A} \circ \negstrong{\sigma'}$
	}
  }
  \Return{$(\mathbf{A}, \Delta)$}
  \caption{\textsc{UnitPropagationTrue}}
  \label{alg:prop-true}
\end{algorithm}

In order to implement efficient propagation, a data

% No Unfounded Set checks -> MBT.

% Explain 2WL with MBT, propagation with MBT (pseudocode of propagation algorithm).

\chapter{Implementation}

\section{Benchmarks}

\chapter{Conclusion}

\section{Related Work}

\section{Conclusion}

\subsection{Related Work}

\backmatter

% Use an optional list of figures.
\listoffigures % Starred version, i.e., \listoffigures*, removes the toc entry.

% Use an optional list of tables.
\listoftables % Starred version, i.e., \listoftables*, removes the toc entry.

% Use an optional list of alogrithms.
\listofalgorithms
\addcontentsline{toc}{chapter}{List of Algorithms}

% Add an index.
\printindex

% Add a glossary.
\printglossaries

% Add a bibliography.
\bibliographystyle{alpha}
\bibliography{thesis}

\end{document}
